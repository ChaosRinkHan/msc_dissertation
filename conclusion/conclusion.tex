\chapter{Conclusion and future works}

%  [what is studied]

%  [to what extend]

%  [how good is the result]

 To conclude, ABC SMC was successfully applied to the models of zebrafish spinal cord regeneration. These models were all ODE equations that described the interactions and effects between cells and proteins in the lesion site. These equations contain four dependent variable and more than 10 model parameters.  Parameter estimation for high-dimensional parameter space is a challenging work for many dynamic system models, as insufficient explore of parameter space can easily lead to local optimal. Using a large population size in ABC SMC in our models could partially relief the concerns in local modes, and give considerable parameter estimates. We found that the first three models could not recover some local features observed in the early times, so further models -- model 4 and 5 -- were proposed. Additional interactions were introduced in the new model, and the results proved that they were helpful in resolving the under-fit.
 
 Also, ABC SMC successfully helped in the model comparison and thus can be used for suggestions of the best model, although there were uncertainties remained in the early generations. In our practice, model 5 won with absolute advantages when the threshold is converged to a small value. The approximated posteriors showed bell-shaped density distribution for most of the parameters (Figure \ref{fig:model5_para} and Figure \ref{fig:para1}), indicating that most parameters were well-inferred in our implementations.

%  ABC SMC implementation is a complex process with respect to both algorithm settings and code development. The ability of parameter inference and robustness of algorithm was firstly studied using synthetic data where the true parameter value was known, and experiments were performed to find proper hyperparameter and implementation settings of ABC SMC. 

% Later inference on the observed data followed these suggestions from the synthetic data experiments and acceptable results were produced. These examples proved that ABC SMC was capable of the inference tasks for the proposed models with high-dimensional parameter space.

 The inference program was mostly run on multi-core machines, and additional scaling-up performance test illustrated the scaling behaviour. When the number of running \verb|Python| processes is less than the number of physical cores (36), a considerable scaling performance was observed. Studies on the variant required numbers of sample revealed that some threshold paths could lead to local optimum and significantly affect the efficiency. A non-linear relationship between the required number of sample and required execution time was also observed, which all made the detailed analysis complicated. 

%  [summary of the findings and advise]

 From this application, some suggestions on ABC SMC implementation could be drawn. The prior distribution and interval of each parameter can affect the inference results significantly, and improper prior cold lead to poor fit. In our cases, log-uniform prior gave better fits than uniform prior. The threshold path and plot of required number of samples in each generation can be used to identify possible local modes; if stuck, it takes much more samples than usual to move out. To avoid this, large population size and a proper epsilon schedule may help \cite{threshold}. For the uncertainties observed in model selection and some duplicated runs with high variance in posteriors, a longer run with more generation could help to identify whether the convergency is consistent such that a reproducible inferred results can be obtained.

%  [thanks]

In terms of the project progress regarding the original project proposal, all the proposed subjects were explicitly studied as planned. We found that code development cost less time than expected, however, implementations and debugging cost more efforts than expected because of both uncertainties in both the inference ability of the algorithm for our problem, and the implementation options and algorithm hyperparameters. Further on our results, we studied some in-depth topics and derived more conclusions, e.g. the relationships between time and sample numbers (Figure \ref{fig:local_modes}) and the movement of posterior concentration (Figure \ref{fig:local_para}). Some interesting further topics were found in the studies, e.g. comparison with other algorithms and implementation optimisations; however, they were not conducted due the insufficient time.

Also, we found that risk analysis and management throughout the project were helpful. Thanks to the planned mitigations and backup strategies, we were able to proceed with our project as planned when interruptions happened. We experienced local laptop broken and unavailability of remote machines, which consequently made our development and implementations harder. We were able to proceed with the project by using backup plans for local development and using Cirrus or ARCHER interchangeably during their downtime.

In terms of the future works, some more comparison with other algorisms could be of our interest. Least square fitting using MCMC \cite{ref:MCMC} and some other exact inference approaches could be performed on the same model and observed data, as the standard deviation at each measurement point is known. By comparing to these methods, differences in the inference results, algorithm robustness and required computational resources could be revealed and possibly help us in choosing an inference algorism that can balance the trade-off between efficiency and a reasonable good `fit'. Also, there are some other implementation methods of ABC SMC worth trying, e.g. CUDA acceleration and \verb|Julia| implementations\footnote[1]{\url{https://github.com/tanhevg/GpABC.jl}} \footnote[2]{\url{https://github.com/marcjwilliams1/ApproxBayes.jl}}.

In the aspect of models, some further models can be studied. If there were some more experimental evidence on the proposed interaction map, we could propose more precise models, or calibrate some terms in our existing models. Also, models of other forms, e.g. PDE and stochastic models, could also be studied if a more complex model is needed for the dynamic system. Simplification is also possible of existing models. Some highly-correlated parameters could be reduced, as there exist some linear relationships between parameters.  

% Beyond this scope, we are interested in the 