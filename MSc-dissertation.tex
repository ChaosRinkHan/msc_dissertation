\documentclass[12pt,a4paper]{report}

\usepackage{epcc}
\usepackage{graphics}

% This example file shows how a thesis can be laid out using Latex. It
% does not use any special local features so should be portable to other
% places.
% 
% When producing draft copies of a thesis you may want to print only
% selected pages of the thesis. To do this use the command
% 
% dvips -f -p 4 -n 3 myfile.dvi | lpr
% 
% where -p 4 means start printing at page 4 (ie the page that will be
% numbered 4, not necessarily the 4th page) and -n 3 means print 3 pages.
% This example will print pages 4, 5 and 6.
% 
% If you want to print the thesis and also save paper you can print more
% than one page on each sheet of paper. Use the command
% 
% dvips -f myfile.dvi | psnup -2 | lpr
% 
% to print 2 pages per sheet. psnup can take values 2, 4, 8, or 9.
%
% To produce a PDF version you can create a PostScript copy first
%
% dvips -f myfile.dvi > myfile.pdf
%
% and then convert it
%
% distill myfile.ps
%
% or you can go straight to PDF
%
% pdflatex myfile
%
% Note that pdflatex expects all included figures to be in PDF too. See
% the includegraphics command below.


% This document contains many cross-references and forward references,
% eg in constructing a table of contents, so Latex may need to be run
% twice to get all the references correct. If you need to run Latex twice
% you may get the warning:
% 
% LaTeX Warning: Label(s) may have changed. Rerun to get cross-refSerences right.


% the following 4 lines are the content of the smallmargins.sty file
% but including them explicitly makes this more portable.
%AC%\oddsidemargin=0.1in
%AC%\topmargin=-0.5in
%AC%\textheight=9in
%AC%\textwidth=6.25in

%AC%\parskip 10pt
%AC%\parindent 0in

\begin{document}

%AC%\pagestyle{myheadings}
%AC%\markright{D.~S.~Henty}

%\title{A Latex thesis example}
%\author{D.~S.~Henty}
%\date{\today}

%\maketitle

\pagenumbering{roman}

\title{Bayesian Inference Using Sequential Monte-Carlo Algorithm for Dynamic System Models}
\author{Chaolin Han}
\date{\today}

\makeEPCCtitle

\thispagestyle{empty}

\vspace{11cm}

\begin{center}

\large{MSc in High Performance Computing}

\large{The University of Edinburgh}

\large{Year of Presentation: 2020}

\end{center}

\newpage

\begin{abstract}
This is the bit where you summarise what is in your thesis.
\end{abstract}

\pagenumbering{roman}

\tableofcontents
\listoftables
\listoffigures

\begin{titlepage}
\vspace*{2in}
% an acknowledgements section is completely optional but if you decide
% not to include it you should still include an empty {titlepage}
% environment as this initialises things like section and page numbering.
\section*{Acknowledgements}

This template is a slightly modified version of the one developed by
Prof. Charles Duncan for MSc students in the Dept. of Meteorology. His
acknowledgement follows:

{\em This template has been produced with help from many former students who
have shown different ways of doing things. Please make suggestions for
further improvements.}

\end{titlepage}

\pagenumbering{arabic}

\chapter{Introduction}
This should contain a description of your project and the problem you
are trying to solve. Where appropriate you should also include
references to work which has already been done on your topic and
anything else which lets you set your work in context.

One of the things you will need to do is to ensure that you have a
suitable list of references.  To do this you should see \cite{ref:lam}
or some other suitable reference.  Note the format of the citation used
here is the style favoured in this department.  Here is another
reference \cite{ref:bloggs} for good measure.

You will also want to make sure you have no spelling or grammatical
mistakes. To help idwentify spelling mistukes you caan use the commands
{\em ispell} or {\em spell}. See the appropriate manual pages. Remember
that spelling mistakes are not the only errors which can occur. Spelling
checkers will not find errors which are, in fact, valid words such as
{\em there} for {\em their}, nor will they find repeated words which
sometimes occur if your concentration is broken when typing. {\bf There
is no substitute for thorough proof reading!}

\chapter{Background}

\section{Zebrafish spinal cord regeneration}
This is just to show how to break things into sections.

Some paragraphs in this demonstration document are here to provide some
padding so that sections last for more than one page to illustrate what
happens on subsequent pages. Note that the page numbering style is usually
different on the first page of a new chapter than on subsequent pages.

Here is a padding paragraph.  Rhubarb.  More rhubarb.  Yet more rhubarb. 
Rhubarb.  More rhubarb.  Yet more rhubarb.  Rhubarb.  More rhubarb.  Yet
more rhubarb.  Rhubarb.  More rhubarb.  Yet more rhubarb.  Rhubarb. 
More rhubarb.  Yet more rhubarb.  Rhubarb.  More rhubarb.  Yet more
rhubarb.  Rhubarb.  More rhubarb.  Yet more rhubarb.  Rhubarb.  More
rhubarb.  Yet more rhubarb.  Rhubarb.  More rhubarb.  Yet more rhubarb. 
Rhubarb.  More rhubarb.  Yet more rhubarb.  Rhubarb.  More rhubarb.  Yet
more rhubarb.  Rhubarb.  More rhubarb.  Yet more rhubarb.  Rhubarb. 
More rhubarb.  Yet more rhubarb.  Rhubarb.  More rhubarb.  Yet more
rhubarb.  Rhubarb.  More rhubarb.  Yet more rhubarb.   Rhubarb.  More
rhubarb.  Yet more rhubarb.   Rhubarb.  More rhubarb.  Yet more
rhubarb.  Rhubarb.  More rhubarb.  Yet more rhubarb.    Rhubarb.  More
rhubarb.  Yet more rhubarb.  Rhubarb.  More rhubarb.  Yet more rhubarb. 

\section{Mathematical modelling}
Some bits are hard.

Here is a padding paragraph.  Rhubarb.  More rhubarb.  Yet more rhubarb. 
Rhubarb.  More rhubarb.  Yet more rhubarb.  Rhubarb.  More rhubarb.  Yet
more rhubarb.  Rhubarb.  More rhubarb.  Yet more rhubarb.  Rhubarb. 
More rhubarb.  Yet more rhubarb.  Rhubarb.  More rhubarb.  Yet more
rhubarb.  Rhubarb.  More rhubarb.  Yet more rhubarb.  Rhubarb.  More
rhubarb.  Yet more rhubarb.  Rhubarb.  More rhubarb.  Yet more rhubarb. 
Rhubarb.  More rhubarb.  Yet more rhubarb.  Rhubarb.  More rhubarb.  Yet
more rhubarb.  Rhubarb.  More rhubarb.  Yet more rhubarb.  Rhubarb. 
More rhubarb.  Yet more rhubarb.  Rhubarb.  More rhubarb.  Yet more
rhubarb.  Rhubarb.  More rhubarb.  Yet more rhubarb.   Rhubarb.  More
rhubarb.  Yet more rhubarb.   Rhubarb.  More rhubarb.  Yet more
rhubarb.  Rhubarb.  More rhubarb.  Yet more rhubarb.    Rhubarb.  More
rhubarb.  Yet more rhubarb.  Rhubarb.  More rhubarb.  Yet more rhubarb. 

\subsection{Hard bits}
You might want to include an equation here:

\begin{equation} \delta N_{\nu} = (\delta N_{\nu})_{ex} + (\delta N_{\nu})_{au} 
\label{equation:delsplit}
% note that the label is optional but allows you to refer to this
% equation later.
\end{equation}

Here is a padding paragraph.  Rhubarb.  More rhubarb.  Yet more rhubarb. 
Rhubarb.  More rhubarb.  Yet more rhubarb.  Rhubarb.  More rhubarb.  Yet
more rhubarb.  Rhubarb.  More rhubarb.  Yet more rhubarb.  Rhubarb. 
More rhubarb.  Yet more rhubarb.  Rhubarb.  More rhubarb.  Yet more
rhubarb.  Rhubarb.  More rhubarb.  Yet more rhubarb.  Rhubarb.  More
rhubarb.  Yet more rhubarb.  Rhubarb.  More rhubarb.  Yet more rhubarb. 
Rhubarb.  More rhubarb.  Yet more rhubarb.  Rhubarb.  More rhubarb.  Yet
more rhubarb.  Rhubarb.  More rhubarb.  Yet more rhubarb.  Rhubarb. 
More rhubarb.  Yet more rhubarb.  Rhubarb.  More rhubarb.  Yet more
rhubarb.  Rhubarb.  More rhubarb.  Yet more rhubarb.   Rhubarb.  More
rhubarb.  Yet more rhubarb.   Rhubarb.  More rhubarb.  Yet more
rhubarb.  Rhubarb.  More rhubarb.  Yet more rhubarb.    Rhubarb.  More
rhubarb.  Yet more rhubarb.  Rhubarb.  More rhubarb.  Yet more rhubarb. 

Here is a padding paragraph.  Rhubarb.  More rhubarb.  Yet more rhubarb. 
Rhubarb.  More rhubarb.  Yet more rhubarb.  Rhubarb.  More rhubarb.  Yet
more rhubarb.  Rhubarb.  More rhubarb.  Yet more rhubarb.  Rhubarb. 
More rhubarb.  Yet more rhubarb.  Rhubarb.  More rhubarb.  Yet more
rhubarb.  Rhubarb.  More rhubarb.  Yet more rhubarb.  Rhubarb.  More
rhubarb.  Yet more rhubarb.  Rhubarb.  More rhubarb.  Yet more rhubarb. 
Rhubarb.  More rhubarb.  Yet more rhubarb.  Rhubarb.  More rhubarb.  Yet
more rhubarb.  Rhubarb.  More rhubarb.  Yet more rhubarb.  Rhubarb. 
More rhubarb.  Yet more rhubarb.  Rhubarb.  More rhubarb.  Yet more
rhubarb.  Rhubarb.  More rhubarb.  Yet more rhubarb.   Rhubarb.  More
rhubarb.  Yet more rhubarb.   Rhubarb.  More rhubarb.  Yet more
rhubarb.  Rhubarb.  More rhubarb.  Yet more rhubarb.    Rhubarb.  More
rhubarb.  Yet more rhubarb.  Rhubarb.  More rhubarb.  Yet more rhubarb. 

\subsection{Even harder bits}
This might be one of the places where you might want to refer to
equation \ref{equation:delsplit}. You will usually need to use the
Latex command twice to make cross-references like this work properly.
The cross-reference information is stored in the {\em .aux} file so don't
delete it.

\subsubsection{Numbering}
You can keep subdividing but eventually you get to a level where
numbering stops. This text is in a subsubsection which is not numbered
by default.


\paragraph{More on numbering:}

This text is in a paragraph which is also not numbered by default and
the ``title'' of the paragraph is not on a separate line.
If you want to increase the depth to which sections are numbered you
should see the section on setting the secnumdepth counter in the manual. 

\section{Bayesian inference}

\chapter{Mathematical modelling}

You might sometimes want to include equations without numbering them.

\begin{displaymath}
E=mc^{2}
\end{displaymath}

You might also want to include diagrams.  The example shows the use of
the special command which allows existing postscript files to be
included.  You would normally keep your figures separate from the text. 
These pictures might be satellite images or postscript output from some
program such as IDL, PV-WAVE, Uniras or xpaint.

Below I create a figure which is centred and stretched to 30\% of the
width of the page \verb+{0.30\hsize}+ and with the height stretched by
the same amount \verb+{!}+ to preserve the aspect ratio. If you omit the
extension (ie .eps, .ps or .pdf) on the file name then latex will pick
up the postscript copy whereas pdflatex will automatically pick up the
PDF version.


\begin{figure}

\begin{center}
\resizebox{0.30\hsize}{!}{\includegraphics{logos/crest_bw}}
\end{center}

\caption{The University Crest}
\label{fig:eucrest}

\end{figure}


% see the man page for dvips for details of the special command which is
% much more powerful than is shown here. It allows offsets in the
% horizontal and vertical and scaling in x and y.

% choosing suitable values for offset and scale can be a tiring matter
% of trial and error.

% note that labels do not need to include a description of the object
% they are labelling but it can be helpful, eg \label{fig:figurename}.

You can use a label on a figure to refer to it later. The university
crest is in \ref{fig:eucrest}. Note that you should not use phrases like
``the figure above'' or ``the following figure'' since Latex may move
the figure relative to the text if it cannot be fitted onto the current page.

\chapter{Implementations and Experiments}

\section{Parameter estimation}
Here are some results.

\section{Model comparison}
Here are some results.

\section{Performance experiments}

The performance experiments are designed to explore the parallel performance of ABC SMC implementations. Usually ABC SMC is a time-consuming and computation intensive task and usually is executed on large clusters. The scheduling strategy, implementation details and many other factors can affect the parallel efficiency.

First experiments are designed to illustrate the scaling-up performance. The program used here is a 

% in practice you would probably keep this in a separate file and use
% the \include{filename} command to insert it here.



\chapter{Results Analysis}

\section{ABC SMC results}

\section{Performance experiments}

\section{Discussions}
This is the place to put your conclusions about your work. You can
split it into different sections if appropriate. You may want to include
a section of future work which could be carried out to continue your
research.

\chapter{Future works}

\chapter{Conclusions}

\appendix
% the appendix command just changes heading styles for appendices.

\chapter{Stuff which is too detailed}

Appendices should contain all the material which is considered too
detailed to be included in the main bod but which is, nevertheless,
important enough to be included in the thesis.

\chapter{Stuff which no-one will read}

Some people include in their thesis a lot of detail, particularly
computer code, which no-one will ever read. You should be careful that
anything like this you include should contain some element of uniqueness
which justifies its inclusion.

\begin{thebibliography}{100}

\bibitem{ref:lam} L.Lamport. {\em 1986 Latex User's Guide
and Reference Manual.} Addison Wesley. pp242.

\bibitem{ref:bloggs} F.Bloggs. {\em 1993 Latex Users do it
in Environments} Int. Journal of Silly Findings. pp 23-29.

\end{thebibliography}


\end{document}

