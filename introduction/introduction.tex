\chapter{Introduction}

 % This should contain a description of your project and the problem you
 % are trying to solve. Where appropriate you should also include
 % references to work which has already been done on your topic and
 % anything else which lets you set your work in context.

 % One of the things you will need to do is to ensure that you have a
 % suitable list of references.  To do this you should see \cite{ref:lam}
 % or some other suitable reference.  Note the format of the citation used
 % here is the style favoured in this department.  Here is another
 % reference \cite{ref:bloggs} for good measure.

%  [background and motivation]

%  [what is done]

While we are interested in the tissue regeneration among different species, mathematical models and techniques can help us to resolve complex interactions and effects of cells and chemicals in a biological system. The objective in this dissertation is an example of complex biology system found in tissue regeneration: zebrafish spinal cord regeneration. Zebrafish has the ability to regenerate its spinal cord after injury; at the lesion site various immune cells are presented and complex interactions are found while they are regulating inflammation and repair.

Our interest lies in the modeling of the regeneration process that happens around the lesion site, especially in the model parameter and comparison of different models. According to available researches \cite{ref:Tsarouchas}, a mathematical model was firstly proposed but the addressed interactions remains to be precisely defined, and quantitatively examined in experiments. However, we would like to explore the the `best fit' of the model which comes with a set of parameter estimates, given the observed data. Also, we wish to compare different it with other alternative models and possibly find the best model to describe the observed data. In these steps, models are expected to retain mathematical and biological significance, where differential equations are suitable to describe this system.

Fo the parameter estimation tasks, Bayesian inference techniques can help in the problems of over-fitting and local optima \cite{ref:abcsysbio}, compared to optimisation-based methods. Moreover, approximate inference methods (Approximate Bayesian Computation, ABC) make likelihood-free inference possible, which is suitable for our task as writing down likelihood as a function expression for the dynamic system can be hard.

Such methods introduce computation-intensive sampling-and-test patterns, where massive number of samples are drawn and followed by complex computations. As the sampling is independent from each other in a certain period, the parallel implementation of the algorithm becomes feasible and is expected to accelerate the parameter inference in a large scale when using HPC resources.

This dissertation mainly focuses on the implementations and experiments of parameter estimation using ABC SMC (sequential Monte-Carlo) method, and discussions on the results. Additionally, performance experiments illustrated the strong-scaling performance of the algorithm. Chapter 2 talks about the background information and theories, Chapter 3 shows how the mathematical models are derived from hypothesis and Chapter 4 describes the ABC SMC implementations and experiments on the models in detail. A brief scaling-up performance experiment is included in Chapter 5, and Chapter 6 summarises the conclusions and states the future works on this topic.
